\expandafter\ifx\csname ifdraft\endcsname\relax
 \begin{document}
\fi

\section{計算}

\subsection{VCO2}

\subsection{STPD係数}

ミキシングチャンバーに取り付けられたセンサーによって測定される換気量は,BTPS(Body Temperature, Pressure, Saturated with water vapor)においての値である.これは慣習として体温37℃,水蒸気飽和,計測環境気圧化における容積値として扱われる.VO2, VCO2は慣習としてSTPD(Standard Temperature, Pressure, Dry),0℃,1気圧の気体標準状態で表記されるため,BTPSを係数を用いてSTPDに換算する必要がある.この係数をSTPD係数と呼ぶ.STPD係数は以下の式で表される.

\begin{equation}
  STPD = \frac{P_B - P_{H2O}}{760} \times \frac{273.15}{273.15 + T}
\end{equation}

ただし,P_B: 気圧,P_{H2O}: 飽和水蒸気圧(\ref{sec:SWVP}で記述する),T: 気温,273.15: 絶対温度である.

\subsubsection{飽和水蒸気圧}
\label{sec:SWVP}

飽和水蒸気圧P_{H2O}は温度によって変化し,Tetens(1930)の式を用いて近似値を求めることができる.温度T[℃]の時の飽和水蒸気圧e(T)[mmHg]は次の式で求められる.

\begin{equation}
  e(T) = 6.1078 \times 10 ^ \frac{7.5T}{(T + 237.3)} \times \frac{760}{1013.25}
\end{equation}

\expandafter\ifx\csname ifdraft\endcsname\relax
  \end{document}
\fi
