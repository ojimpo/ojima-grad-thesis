\expandafter\ifx\csname ifdraft\endcsname\relax
 \begin{document}
\fi

\section{考察}

\subsection{酸素摂取量の変化に関する考察}

低強度と高強度における酸素摂取量の変化(\ref{fig:light_hard_vo2})を見ると,高強度で低強度より酸素摂取量が大きくなっているのが分かる.また,開始6分時点に0Wから70Wに漸増する低強度と,開始3分時点に40Wから70Wに漸増する設定パワーによる酸素摂取量の増加のタイミングの違いが確認できる.低強度,高強度いずれの場合でもその後3分ごとに30Wずつ漸増していくが,最初の1分程度では低強度と高強度で同等の割合で増加を見せるのに対し,以降の増加の割合は設定パワーが高い高強度では低強度よりも大きくなっていることが確認できる.

\subsection{酸素負債に関する考察}

低強度における酸素摂取量とパワーの比較(\ref{fig:light_vo2_power})では負荷の漸増によって増加し15分時点で最大となった酸素摂取量が,漸減を始める15分時点以降でも増加時と同じような波形で減少しているのが分かる.一方で,高強度における酸素摂取量とパワーの比較(\ref{fig:hard_vo2_power})では,負荷を漸増していく15分時点までは酸素摂取量と同様の傾向で増加していくが,漸減を始める15分時点以降はパワーの波形を遅れてなぞるように減少していることが分かる.

高強度にのみ現れている負荷漸減中の酸素摂取量の増加は,15分時点までの負荷漸増に対応するために発生した酸素借を補うための酸素負債が発生している状態だと考えられる.今回は酸素負債の発生を抑えるための実験プロトコルを設定したが,実際には酸素負債が発生してしまったということである.また,この結果から今回の実験において,酸素負債が発生する境界となるパワー値は,低強度の最大設定パワーである140Wから高強度の最大設定パワーである170Wの間にあるということが考えられる.

また,増加に比べてゆるやかに減少していく変化の傾向は,高強度における酸素摂取量と心拍数の比較(\ref{fig:hard_vo2_hr})においても同様に見られる.これは,酸素が血液によって搬出されているため,酸素摂取量が心拍数と同様の変化を示していると考えられる.

\subsection{酸素摂取量のピーク位置に関する考察}

低強度と高強度における酸素摂取量の比較(\ref{fig:light_hard_vo2})より,低強度,高強度いずれの場合においても,酸素摂取量の最大値位置は15分付近で一致している.また,パワー,心拍数との比較においても全て最大値は一致している.一方で,低強度における8分付近や高強度における21分付近などに,パワーと心拍数には現れないピークが出現している.低強度(\ref{fig:light_vo2_power}),高強度(\ref{fig:hard_vo2_power})それぞれのパワーとの比較から見ると,設定パワーが切り替わるタイミングでこれらのピークが出現していることが分かる.同様の傾向は,何度か行った予備実験でも頻繁に見られた.

\subsubsection{流量計の不安定さ}

酸素摂取量の変化は大部分で換気量に追随するが,今回使用した流量計が出力する流量が不安定であることがこれらのピークが出現する原因だと考えられる.同様の傾向は,流量計パルス数からの空気流量を換算する係数を測定する実験(\ref{sec:measuring_coefficient})を行った際にも,基本的には同じパルス数が測定されるのに,時折大きく外れたパルス数が測定されることからも見られていた.今回使用した流量計は本来水流を測定するために設計されているため,空気の流量を測定する際の性能は保証されていない.今回は安価に製作するために水流計を換気量計としてそのまま使用したが,正確な換気量を得るには,羽根車の形状の変更や,軸受部の潤滑の改善など水流計の改良が必要であると考えられる.

\subsection{流量計の改善}

今回流量計として使用した水流系は,測定値が不安定であることや,流路抵抗の大きさなど多数の問題が存在する.これらの問題を根本的に解消するためには,流量計をまったく別の部品を用いて取り換えることが望ましい.交換する流量計に求められる性能には,今回の知見から以下の点が挙げられる.

\begin{itemize}
  \item 大きな流路径(直径35mm以上が望ましい)を持つ.
  \item 使用状況に左右されずにいつも安定した測定値を出力する.
  \item 取付角度に制限を受けない.
\end{itemize}

\subsubsection{タービン式水流計の校正方法}

流量計は本来,時間あたりの容量をタービンが回転した際に出力するパルスの数で測定するために設計されている.今回はこれを一定量を不定の時間で流した際のパルスの数が一定になると仮定した上で,一定量あたりのパルス数を実測によって求めて流量計として使用したことになる.実際に測定を行って,やはり流量計の精度に不満が出たのは,やはりこの校正方法に問題があったものと思われる.

今回使用した流量計をそのまま使用する場合,最も良い校正方法は一定流量で空気を流し続ける装置を使って,時間あたりのパルス数を求めることで流量計を校正することであると考えられる.

実際にはこのような装置を用意するのは難しいので,今回の校正方法を改善する形で,今回使用した300mLのシリンジよりも大きな容量のシリンジを用意して,毎回正確に等しい時間でピストンを押せるような仕組みで一定量あたりのパルス数を測定する方法を考える必要がある.

\subsection{最大作業の測定}

今回の装置は,センサーの使用や性能の問題から,最大作業時の呼吸代謝の測定は想定しないこととして製作した.しかし,最大酸素摂取量の測定や日々のトレーニングにおける進捗確認など,最大作業の測定が可能になる事で装置の使い道は大きく広がることが想像される.ここでは最大作業の測定を可能にするために可能な装置の改善について考察する.

\subsubsection{二酸化炭素濃度の測定範囲}

今回使用した二酸化炭素センサーは入手可能なものの中から最も測定範囲が広いものを選んだとはいえ,最大4\%の測定範囲では,今回の実験プロトコル以上程度の運動強度で運動を行うとすぐに二酸化炭素濃度が4\%に張り付いてしまうことが確認された.最大作業の測定を行うためにはまず最初に9\%程度までの測定範囲を持つ二酸化炭素センサーが必要である.

\subsubsection{流路抵抗の大きさ}

実験中,低強度,高強度のいずれの場合でも,設定パワーが110Wを超えたあたりから,マスク越しには思う存分に呼吸ができず息苦しさを感じることが確認できた.このような状態においては,わずかに外気が入るマスクの隙間から呼吸をしようとするような状態になるため,正確な酸素摂取量の測定にも悪影響を与えると考えられる.呼吸代謝測定装置において最大作業の測定を行うためには直径35mm以上の流路系が必要とされるが,今回の装置では,水流計YF-S201の流路内に直径が9mmまで小さくなっている箇所がある他,使用した逆流防止弁の流量が小さいことなど,流路抵抗となっていると思われる箇所が多数存在する.これらの抵抗を取り除く事で呼吸が楽になり,最大作業の測定が可能になると思われる.

\subsection{酸素センサーの使い勝手}

今回の装置では,通常なら1万円程度のセンサーの代わりに,酸素センサーとして空気亜鉛電池を酸素センサーとして利用するキットを使用した.空気亜鉛電池の構造的に,ミキシングチャンバー内で呼気酸素濃度を測定する用途には不適であると思われたが,出力値は安定しており,以上値を出すことは無かった.ただ.マイコン側のADコンバーターのノイズ問題があり,それを除去するためのデジタルフィルターにより,信号に遅れが出てしまうので,それが測定値に影響を与えている可能性は否定できない.また,本来呼気酸素濃度は16\%程度まで低下することがあるはずだが,今回は最低でも18\%程度しか見られなかったので,何らかの理由で正しく測定できていなかった可能性は存在する.

また,空気亜鉛電池は容量が小さく寿命が短いため測定ごとに可変抵抗を調整して校正を行う必要があったが,デジタルフィルターによる信号の遅れもあって,適正な校正値に調整するのは何度も適正値を通り過ぎてしまって手間だった.構造的にデジタル的に校正を行うことはできないが,このセンサーを酸素センサーとして利用するためには,校正をしやすくするためにデジタルフィルターをかける必要がないように,低ノイズの高性能外部ADコンバーターを利用するべきである.

\subsection{測定データの利用}

\subsubsection{オープンソースハードウェア化}

現在存在する欠点を改善し,既存の呼吸代謝測定装置との比較などによって精度が保証できる位のレベルの完成を見た場合,ソフトウェアのコード,3Dプリント部品のデータをオープンソースで公開することで,オープンソースの呼吸代謝測定装置を作りたいという構想を抱いている.

\subsection{ソフトウェアの改善}

体重あたりの酸素摂取量を求める際に重要な体重は,今回のプログラムではArduinoスケッチ内に定数として書き込んだ.実際に測定を行ってみると,体重を変更するたびに定数を変更してコンパイルし直し,M5Stack Core2に書き込む必要があり,非常に手間であることが分かった.そこで,頻繁に書き換えが必要になる体重や身長などの数値,またはWi-FIのパスワードなどは,Micro SDカード内の設定ファイルから読み込む方式として,変更を容易にするのが良いと考えられる.

\expandafter\ifx\csname ifdraft\endcsname\relax
  \end{document}
\fi
