\expandafter\ifx\csname ifdraft\endcsname\relax
 \begin{document}
\fi

\section{考察}

\subsection{酸素摂取量の変化に関する考察}

低強度と高強度における酸素摂取量の変化(\ref{fig:light_hard_vo2})を見ると,設定パワーが同じ開始時点では近かった酸素摂取量が,高強度の場合ではより大きな割合で増加していることが分かる.低強度よりも高強度において」

\subsection{酸素負債に関する考察}

低強度における酸素摂取量とパワーの比較(\ref{fig:light_vo2_power})では負荷の漸増によって増加し15分時点で最大となった酸素摂取量が,漸減を始める15分時点以降でも増加時と同じような波形で減少しているのが分かる.一方で,高強度における酸素摂取量とパワーの比較(\ref{fig:hard_vo2_power})では,負荷を漸増していく15分時点までは酸素摂取量と同様の傾向で増加していくが,漸減を始める15分時点以降はパワーの波形を遅れてなぞるように減少していることが分かる.
高強度にのみ現れている負荷漸減中の酸素摂取量の増加は,15分時点までの負荷漸増に対応するために発生した酸素借を補うための酸素負債が発生している状態だと考えられる.よって,今回の実験において,酸素負債が発生する境界となるパワー値は低強度の最大設定パワーである140Wから高強度の最大設定パワーである170Wの間にあるということが考えられる.

\subsection{酸素摂取量のピーク位置に関する考察}

低強度と高強度における酸素摂取量の比較(\ref{fig:light_hard_vo2})より,低強度,高強度いずれの場合においても,酸素摂取量の最大値位置は15分付近で一致している.また,パワー,心拍数との比較においても全て最大値は一致している.一方で,低強度における8分付近や高強度における21分付近などに,パワーと心拍数には現れないピークが出現している.低強度(\ref{fig:light_vo2_power}),高強度(\ref{fig:hard_vo2_power})それぞれのパワーとの比較から見ると,設定パワーが切り替わるタイミングでこれらのピークが出現していることが分かる.同様の傾向は,何度か行った予備実験でも頻繁に見られた.

\subsubsection{流量計の不安定さ}

酸素摂取量の変化は大部分で換気量に追随するが,今回使用した流量計が出力する流量が不安定であることがこれらのピークが出現する原因だと考えられる.今回使用した流量計は本来水流を測定するために設計されているため,空気の流量を測定する際に十分な軸潤滑がなされておらず,
水は空気の1000倍程度を持つため,同量の流量の空気と水が流量計を流れる場合では


\subsection{換気量センサーのボトルネック}

\subsection{測定データの利用}

\subsubsection{オープンソースハードウェア化}

\subsection{装置に関する考察}

\subsubsection{流量計の不安定さ}

今回流量計として使用した水流センサーYF-S201は

今回使用した酸素センサー,A-5Sに使用する酸素亜鉛電池PR44は電圧を維持する時間が短く

体重と身長をSDカードに書くようにすればよかった

\expandafter\ifx\csname ifdraft\endcsname\relax
  \end{document}
\fi
