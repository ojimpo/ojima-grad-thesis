\expandafter\ifx\csname ifdraft\endcsname\relax
 \begin{document}
\fi

\section{考察}

\subsection{酸素摂取量の変化に関する考察}

低強度と高強度における酸素摂取量の変化(\ref{fig:light_hard_vo2})を見ると,高強度で低強度より酸素摂取量が大きくなっているのが分かる.また,開始6分時点に0Wから70Wに漸増する低強度と,開始3分時点に40Wから70Wに漸増する設定パワーによる酸素摂取量の増加のタイミングの違いが確認できる.低強度,高強度いずれの場合でもその後3分ごとに30Wずつ漸増していくが,最初の1分程度では低強度と高強度で同等の割合で増加を見せるのに対し,以降の増加の割合は設定パワーが高い高強度では低強度よりも大きくなっていることが確認できる.

\subsection{酸素負債に関する考察}

低強度における酸素摂取量とパワーの比較(\ref{fig:light_vo2_power})では負荷の漸増によって増加し15分時点で最大となった酸素摂取量が,漸減を始める15分時点以降でも増加時と同じような波形で減少しているのが分かる.一方で,高強度における酸素摂取量とパワーの比較(\ref{fig:hard_vo2_power})では,負荷を漸増していく15分時点までは酸素摂取量と同様の傾向で増加していくが,漸減を始める15分時点以降はパワーの波形を遅れてなぞるように減少していることが分かる.

高強度にのみ現れている負荷漸減中の酸素摂取量の増加は,15分時点までの負荷漸増に対応するために発生した酸素借を補うための酸素負債が発生している状態だと考えられる.今回は酸素負債の発生を抑えるための実験プロトコルを設定したが,実際には酸素負債が発生してしまったということである.また,この結果から今回の実験において,酸素負債が発生する境界となるパワー値は,低強度の最大設定パワーである140Wから高強度の最大設定パワーである170Wの間にあるということが考えられる.

また,増加に比べてゆるやかに減少していく変化の傾向は,高強度における酸素摂取量と心拍数の比較(\ref{fig:hard_vo2_hr})においても同様に見られる.これは,酸素が血液によって搬出されているため,酸素摂取量が心拍数と同様の変化を示していると考えられる.

\subsection{酸素摂取量のピーク位置に関する考察}

低強度と高強度における酸素摂取量の比較(\ref{fig:light_hard_vo2})より,低強度,高強度いずれの場合においても,酸素摂取量の最大値位置は15分付近で一致している.また,パワー,心拍数との比較においても全て最大値は一致している.一方で,低強度における8分付近や高強度における21分付近などに,パワーと心拍数には現れないピークが出現している.低強度(\ref{fig:light_vo2_power}),高強度(\ref{fig:hard_vo2_power})それぞれのパワーとの比較から見ると,設定パワーが切り替わるタイミングでこれらのピークが出現していることが分かる.同様の傾向は,何度か行った予備実験でも頻繁に見られた.

\subsubsection{流量計の不安定さ}

酸素摂取量の変化は大部分で換気量に追随するが,今回使用した流量計が出力する流量が不安定であることがこれらのピークが出現する原因だと考えられる.同様の傾向は,流量計パルス数からの空気流量を換算する係数を測定する実験(\ref{sec:measuring_coefficient})を行った際にも,基本的には同じパルス数が測定されるのに,時折大きく外れたパルス数が測定されることからも見られていた.今回使用した流量計は本来水流を測定するために設計されているため,空気の流量を測定する際の性能は保証されていない.今回は安価に製作するために水流計を換気量計としてそのまま使用したが,正確な換気量を得るには,羽根車の形状の変更や,軸受部の潤滑の改善など水流計の改良が必要であると考えられる.

\subsection{換気量センサーのボトルネック}

\subsection{最大作業の測定}



\subsection{測定データの利用}

\subsubsection{オープンソースハードウェア化}

\subsection{装置に関する考察}

\subsubsection{流量計の不安定さ}

今回流量計として使用した水流センサーYF-S201は

今回使用した酸素センサー,A-5Sに使用する酸素亜鉛電池PR44は電圧を維持する時間が短く

体重と身長をSDカードに書くようにすればよかった

\expandafter\ifx\csname ifdraft\endcsname\relax
  \end{document}
\fi
