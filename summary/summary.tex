\expandafter\ifx\csname ifdraft\endcsname\relax
 \begin{document}
\fi

\section*{卒業論文 2021年度(令和3年度)\\汎用センサーを用いた呼吸代謝測定装置の自作}
\subsection*{論文要旨}
\noindent %先頭を改行しないという指示
%ここに論文要旨を書く.

現在,スポーツ中の各種データを取得できるデバイスはアマチュアレベルでも広く普及しているが,呼気分析によって得られるデータを取得できるデバイスは普及しているとは言いがたい.呼気分析によって最大酸素摂取量(\.{V}_{O_2max})や消費エネルギーの高精度な推定など,より有用なデータの取得が可能になる.本研究では入手性が高い安価な汎用センサーを用いて,個人レベルで利用できる呼気分析を行うことができるデバイス(呼吸代謝測定装置)を製作し,その精度や有用性を検証する.

\\
\\
キーワード\\
1.センサー 2.呼気分析 3.インドアスポーツ 4.呼吸代謝測定装置 5.マイコン\\
\begin{flushright} %右寄せをするという指示
  慶應義塾大学 環境情報学部\\
  尾島航基
\end{flushright}

\expandafter\ifx\csname ifdraft\endcsname\relax
  \end{document}
\fi
