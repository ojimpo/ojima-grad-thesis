\expandafter\ifx\csname ifdraft\endcsname\relax
 \begin{document}
\fi

\section*{卒業論文 2021年度(令和3年度)\\マイコンと汎用センサーを用いた\\安価なリアルタイム呼吸代謝測定装置の製作}
\subsection*{論文要旨}
\noindent %先頭を改行しないという指示
%ここに論文要旨を書く.

現在,各種データをスポーツ中にリアルタイムに取得できるデバイスが普及しているが,個人レベルで入手可能な呼気分析による各種データをリアルタイムで取得できる安価なデバイスは普及していない.呼気分析によって酸素摂取量(\.{V}O_2Max)や消費エネルギーの高精度な推定など,より有用なデータの取得が可能になれば,個人でのより定量的な能力測定や,COVID-19によって注目が高まるインドアトレーニングなど,様々な場面での利用が期待できる.本研究では入手が容易な安価なマイコンと汎用センサーを用いて,個人レベルで利用できるリアルタイム呼気分析を行うことができるデバイス(呼吸代謝測定装置)を製作し,その精度や有用性を検証する.

\\
\\
キーワード\\
1.ウェアラブル 2.呼気分析 3.酸素摂取量 4.リアルタイム 5.センシング\\
\begin{flushright} %右寄せをするという指示
  慶應義塾大学 環境情報学部\\
  尾島航基
\end{flushright}

\expandafter\ifx\csname ifdraft\endcsname\relax
  \end{document}
\fi
