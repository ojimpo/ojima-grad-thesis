\expandafter\ifx\csname ifdraft\endcsname\relax
 \begin{document}
\fi

\section{緒言}

\subsection{スポーツセンシングの現状と呼気分析}

近年,小型軽量かつ安価なデバイスでスポーツ中の様々なデータを測定・記録することが可能になってきた.特に運動時間が長く激しい動きが少ないことで運動中の測定が行いやすい持久系スポーツにおいては,アマチュアレベルでも様々なデータを記録し、トレーニングに活用するということが行われるようになっている.例として,GPSなどを用いた移動距離やスピード,加速度センサーなどを用いたランニングにおけるピッチやストライド,サイクリングにおけるペダル回転数,また皮膚に取り付けるセンサーで測定する心拍数や血中酸素濃度(SpO_2)などが挙げられる.

一方で,.呼気を採集・分析することによって得られるデータを測定するデバイスは学術・産業用が存在するのみで,個人レベルで使用できる安価なものが普及するには至っていない.これによって得られる最大酸素摂取量(\.{V}_{O_2max})の直接測定や,酸素摂取量と二酸化炭素摂取量を用いた消費エネルギーの高精度な推定などを行うことが可能になれば,そのデータをより効率的なトレーニングや安全な運動に役立てることが可能になる.そこで,本研究では個人レベルで利用できる呼気分析を行うことができるデバイス(呼吸代謝測定装置)を入手性が高い安価な汎用センサーを用いて製作する.

また,2020年に出現し現在も世界中で猛威を振るっているCOVID-19によって,スポーツの在り方も変化を受けている.大人数で集まって行うようなトレーニングが行えなくなったほか,自宅などに居ながらにしてインターネット経由で世界中の人々と競技ができるインドアスポーツは急速な盛り上がりを見せている.本研究はそのような新たなスポーツの在り方に役立つ新たなデバイスを製作することを目的としている.

\expandafter\ifx\csname ifdraft\endcsname\relax
  \end{document}
\fi
