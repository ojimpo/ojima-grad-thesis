\expandafter\ifx\csname ifdraft\endcsname\relax
 \begin{document}
\fi

\section{緒言}

\subsection{スポーツセンシングの現状と呼気分析}

近年,小型軽量かつ安価なウェアラブルデバイスでスポーツ中の様々なデータをリアルタイムに測定・記録することが可能になった.特に持久系スポーツにおいては,運動時間が長く激しい動きが少ないことで運動中の測定が行いやすいことから,アマチュアレベルでも様々なデータを記録し、トレーニングに活用することが普及している.例として,ランニングやサイクリングにおける,GPSを用いた移動距離やスピードの測定や,ランニングやスイミングにおける加速度センサーなどを用いたフォームの分析,サイクリングにおける歪みセンサーを用いたパワー測定など,現在私たちがリアルタイムに測定が可能な項目は非常に多岐にわたる.

一方で,研究室レベルでは古くから行われてきた,呼気分析によって得られる各種データを測定するデバイスは個人レベルで使用できる安価なものが普及するには至っていない.これによって得られる最大酸素摂取量(\.{V}_{O_2max})の直接測定や,酸素摂取量と二酸化炭素摂取量を用いた消費エネルギーの高精度な推定などを行うことが可能になれば,そのデータをより効率的なトレーニングや安全な運動に役立てることが可能になる.そこで,本研究では個人レベルで利用できる呼気分析を行うことができるデバイス(呼吸代謝測定装置)を入手性が高い安価な汎用センサーを用いて製作する.

また,2019年に出現し現在も世界中で猛威を振るっているCOVID-19によって,スポーツの在り方も変化を受けている.大人数で集まって行うようなトレーニングが行えなくなったほか,自宅などに居ながらにしてインターネット経由で世界中の人々と競技ができるインドアスポーツは急速な盛り上がりを見せている.本研究はそのような新たなスポーツの在り方に役立つ新たなデバイスを製作することを目的としている.

また,2019年に出現し現在も世界中で猛威を振るっているCOVID-19によって,感染防止からマスク型の装置を複数人で共用することが不可能になり,従来の高価な呼吸代謝測定装置を研究室等で使用することが難しくなってきた.そのような状況の中で,個人で安価に製作することのできる呼吸代謝測定装置は価値を発揮するのではないかと考える.

\subsection{インドアスポーツの出現とCOVID-19}

\subsection{既存の呼吸代謝測定装置}

\subsection{リアルタイム呼吸代謝測定}

\subsection{今回の研究の目標}


\expandafter\ifx\csname ifdraft\endcsname\relax
  \end{document}
\fi
