\expandafter\ifx\csname ifdraft\endcsname\relax
 \begin{document}
\fi

\section{緒言}

\subsection{スポーツセンシングの現状}

近年,小型軽量かつ安価なウェアラブルデバイスでスポーツ中の様々なデータをリアルタイムに測定・記録することが可能になった.特に持久系スポーツにおいては,激しい動きが少なく運動中の測定が行いやすいことから,現在では様々なデータを記録し、トレーニングに活用することが普及している.例として,ランニングやサイクリングにおける,GPSを用いた移動距離やスピードの測定や,ランニングやスイミングにおける加速度センサーなどを用いたフォームの分析,サイクリングにおける歪みセンサーを用いたパワー測定など,現在私たちがリアルタイムに測定が可能な項目は非常に多岐にわたる.

\subsection{呼気分析の現状}

一方で,研究室レベルでは古くから行われてきた呼気分析によって得られる各種データを測定するデバイスは,個人レベルで使用できる安価なものが普及するには至っていない.それにも関わらず,酸素摂取量を推定することによって得られる最大酸素摂取量や運動強度,消費エネルギーの推定などの指標は多くの人々が利用するデバイスで多くの人が利用するという状況になっている.酸素摂取量などの呼気分析が現在普及しているデバイスのように気軽に利用できるようになれば,そのデータをより効率的なトレーニングや安全な運動に役立てることが可能になるだろう.そこで,本研究では個人レベルで利用できる呼気分析を行うことができるデバイス(呼吸代謝測定装置)を,安価なマイコンと汎用センサーを用いて製作する.

また,2019年に出現し現在も世界中で猛威を振るっているCOVID-19によって,スポーツの在り方も変化を受けている.大人数で集まって行うようなトレーニングが行えなくなった一方で,自宅などに居ながらにしてインターネット経由で世界中の人々と競技ができるインドアスポーツは急速な盛り上がりを見せている.移動を伴わないインドアスポーツはスポーツセンシングとの相性も良く,呼気分析の需要も今後高まっていくことだろう.本研究は,そのような新たなスポーツの在り方に役立つ新たなデバイスを製作することを目的とする.

\subsection{既存の呼吸代謝測定装置}

参考までに,現在日本で購入が可能な呼吸代謝装置とその価格を表\ref{tb:existing_rmmd}に示す.以下の製品はいずれも株式会社フォーアシストが国内で販売している製品で,価格は同社の税抜販売価格である.

\begin{table}[]
\begin{center}
  \caption{既存の呼吸代謝測定装置}
  \label{tb:existing_rmmd}
  \begin{tabular}{|l|l|l|}
  \hline
  製品名              & 備考              & 価格          \\ \hline
  VO2Master        & マスク型デバイスのみで使用可能 & ¥1,280,000- \\ \hline
  Cardio Coach PRO & ミキシングチャンバーが必要   & ¥4,200,000- \\ \hline
  Cardio Coach MAX & ミキシングチャンバーが必要   & ¥3,600,000- \\ \hline
  MetaCheck        & 安静時代謝測定用        & ¥1,300,000- \\ \hline
  \end{tabular}
\end{center}
\end{table}

\subsection{本研究の目標}

今回製作する装置は,呼気分析による呼吸代謝の測定を小型のマイコン本体のみで行い,測定したデータの記録・表示までをその場で確認できるものとする.これをリアルタイムでの呼吸代謝測定とし,この機能を備えた装置を製作することを目指す.

また,装置本体の価格は,表\ref{tb:existing_rmmd}に見られるような既存の装置の100分の1程度を目標とする.


\expandafter\ifx\csname ifdraft\endcsname\relax
  \end{document}
\fi
