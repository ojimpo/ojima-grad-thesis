\expandafter\ifx\csname ifdraft\endcsname\relax
 \begin{document}
\fi

\section{製作}

\subsection{マスク}

\subsubsection{呼気収集マスク}

呼気収集マスクには,仰木研究室内で以前に製作されたマスクを使用した.このマスクは,アクリル板を組み合わせて顔に合うような形状を構成し,吸気及び呼気用の通気口を設けた物である.顔に当たる部分にはウレタンフォーム製のクッションを取り付け,頭に取り付けられるように市販のガスマスクから流用したバンドが取り付けてある.

\subsubsection{逆止弁}

\subsection{換気量計}

\subsubsection{測定方式}

ミキシングチャンバー方式で換気量を測定するため,マスクの呼気方向にのみ解放される弁の先に取り付けた流量計で換気量を測定する

今回はマイコンなどに接続する水量計として安価に市販されているタービン流量計を流量計に用いた.YF-S201という名称で販売されているもので,タービンの回転数をホール素子センサーによって測定するものである.

呼気のような微小な気体の流量を測定するための流量計には,差圧流量計,超音波流量計,タービン流量計などが用いられる.

差圧流量計は,流路内に絞り機構を設置し,その前後に設置した圧力計から得られる圧力差から流量を測定する方式である.絞り機構としてオリフィスプレートを用いたものはオリフィス流量計と呼ばれる.
超音波流量計は流路内を流れる流体に超音波を照射することで流量を測定する方式である.
タービン流量計は流路にタービンを設置し,流体によって回転するタービンの回転数によって流量を測定する方式である.

%差圧流量計は,流路内に絞り機構を設置し,その前後に設置した圧力計から得られる圧力差から流量を測定する方式である.絞り機構としてオリフィスプレートを用いたもの(オリフィス流量計)はCardioCorchやVO2000などの既存の呼吸代謝測定装置で使用されている.

%超音波流量計は流路内を流れる流体に超音波を照射することで流量を測定する方式である.高精度が特徴であり,NASAが開発したPUMAに使用されている.

既存の呼吸代謝測定装置の流量計には主に先述の三方式が用いられるが,今回は水流センサーとして汎用的に安価に入手が可能であるということでタービン流量計を用いた.

\subsubsection{信号処理}

タービン流量計は流路が気体を流れていない時間も数秒間タービンが空転する.よって,タービンが空転しているだけの時間に測定される流量はデータから除外する必要がある.
YF-S201が出力する信号は高周波のノイズ成分があるため,これをハイパスフィルターを用いて除外する必要がある.今回は計算の容易さからRCフィルターを用いた.

\subsection{酸素センサー}

\subsubsection{空気亜鉛電池式センサー}

今回は株式会社ピーバンドットコムから発売されている「実習用酸素センサキット A-5S」(以下「A-5S」)を酸素センサーとして使用した.A-5Sは,補聴器など用に汎用的に使用される空気亜鉛電池をセンサーとして使用し,空気亜鉛電池の出力電圧から酸素濃度を測定するというセンサーで,組み立てキットとして1000円程度で購入が可能である.構造は単純で空気亜鉛電池に固定抵抗と可変抵抗を接続したというものである.キャリブレーションは大気中の酸素濃度20.84\%とに合わせて出力電圧が20.84mVになるように可変抵抗を調整して行う.空気亜鉛電池の電圧の低下から,長時間連続での測定は困難である.

従来,呼吸代謝測定装置の酸素センサーにはガルバニ電池式センサーが多くの場合で使われてきた.この方式は高精度であるが,酸素濃度に応じて電圧を出力することで酸素濃度を測定するためのガルバニ電池が10000円以上と高価であるため,安価に入手できる空気亜鉛電池をセンサーとして用いたA-5Sを使用した.

\subsubsection{信号処理}

A-5Sが出力する電圧は酸素濃度21\%時に21mVと非常に微弱である.この電圧を今回使用したマイコン,M5Core2(ESP32)の12bit ADコンバーター(0-3.3V, 4096段階)で測定するために,アペアンプを使用して増幅した.オペアンプには,単電源のフルスイングオペアンプNJM2732Dを用いた.非反転増幅回路を用いてA-5Sの出力電圧を101倍に増幅し,酸素濃度21\%時に2.1V程度に増幅することで測定精度を高めている.

(回路図)

\subsection{二酸化炭素センサー}

二酸化炭素濃度を測定するセンサーにはMH-Z19Bを用いた.このセンサーはNDIR方式(非分散型赤外線吸収方式)を用いて二酸化炭素の濃度を測定する.NDIR方式は,それぞれのガスが持つ特有の吸収波長領域を利用し,特定のガスのみを対象ガスに変化を及ぼすことなく濃度を測定することができるガス濃度の測定方式である\cite{whats_ndir}.MH-Z19Bは,NDIR方式の二酸化炭素濃度センサーの中でも2000-5000円程度で比較的容易に入手できるものである.

MH-Z19Bはコマンドを送信することで二酸化炭素濃度をppm単位で容易に取得することが可能である.今回はArduino用のライブラリを用いてppm単位の二酸化炭素濃度を取得し,\%単位に変換してVCO_2の計算に使用している.

\subsection{温度計・湿度計・温度計}




\subsection{センサー値の計算}

\subsubsection{計算用マイコン}

\subsubsection{プログラム}

\expandafter\ifx\csname ifdraft\endcsname\relax
  \end{document}
\fi
