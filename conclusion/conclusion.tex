\expandafter\ifx\csname ifdraft\endcsname\relax
 \begin{document}
\fi

\section{結言}

本研究の目的は,リアルタイムでの呼吸代謝測定が行える装置を既存の100分の1程度の価格で製作することであった.

結果として,原理を満たすセンサーを組み合わせ,マイコンで処理・表示を行うことで,リアルタイムで酸素摂取量などの呼気分析が利用できる装置を製作することができた.しかし,装置の実用性としては,実際の利用用途では重要になると考えられる最大作業における測定に対応できないことや,安価な汎用センサーを使用するゆえの測定範囲,精度,安定性などの問題など,測定機材としては多くの問題を残すものである.

特に,水量計を転用した流量計は,精度,流路径の小ささなど,いくつかの問題の原因となっていると思われる部分であり,今回した装置を実用するにあたっては真っ先に改善するべき部分であると考えられる.

また,今回製作した装置を実際に使ってみた感想として,装置自体は安価であるものの,呼気収集用のマスクを顔に装着する必要があること,装置を使用する間は少なかれ呼吸が苦しくなるなど,呼吸代謝測定装置を一般のスポーツ愛好家が使う上での克服できない課題があることを感じる結果となった.

しかし,今までほとんど利用することができなかった呼気分析という方法を用いて測定した値を見ながら運動をするという体験は,間違いなく新鮮さを感じるものであると言える.

本研究では,安価なマイコンと汎用センサーを用いた呼吸代謝測定装置を製作することは可能だという結果を得ることができた.

\expandafter\ifx\csname ifdraft\endcsname\relax
  \end{document}
\fi
