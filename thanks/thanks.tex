\expandafter\ifx\csname ifdraft\endcsname\relax
 \begin{document}
\fi

\section*{謝辞}
%ここに謝辞を書く

本研究は新型コロナウィルス感染症の影響を多大に受け,最初のテーマ選定の段階から最後の実験まで,ほとんどの作業を自宅で行うことになりました.この経験は,研究室で先生や先輩,友人たちとディスカッションを重ねて研究を進める例年とは大きく異なるものになったことと思います.

そのような状況において,私に期待をかけてテーマを与えてくださり,オンラインで熱心なご指導を頂いた仰木裕嗣先生に感謝の意を表します.ハードウェア・ソフトウェア,電子工作,3Dプリンター,運動生理学,スポーツと私の大好きな事を大学生活の集大成となる卒業論文のテーマとできたことを嬉しく思います.この謝辞を書いている時点で,これからもしばらく本研究に協力していただくことになりましたが,どうぞよろしくお願いします.

研究室の先輩である石塚辰郎さんには,提出数時間前というタイミングにも関わらず的確無比な添削をしていただきました.本卒業論文が体裁を保つことができたのは大部分が石塚さんのおかげです.思えば学部1年生の頃に初めて仰木先生の授業を受講した時からTAとして,研究室に入ってからは先輩として,研究のいろはを一番近い立場から私に示していただいたこと,深く感謝いたします.

本研究に使用した3Dプリンター,フィラメントは慶應SFC学会の「Covid-19対応研究活動支援特別募集」の助成を受けて購入させて頂いたものです.これらの機材で高精度な部品を高速に製作することができなければ,本研究を進めることは非常に困難だったと思います.深く感謝いたします.

私の現在のアルバイト先である株式会社グロータックには,実験に使用したローラー台などの機材を貸与していただきました.追い込みの時期には,1ヶ月近くの休みをいただいたことを感謝します.

九州と神奈川にいる頼れる友人たちには,一人では自信が持てない研究の進め方や計算方法についての相談に対し,素晴らしいスピード感で回答を頂けてお世話になりました.数えきれない普段の雑談の内容が本研究に活きることも多分にあったと思います.

最後に,同じ屋根の下で暮らし,あらゆる支援をくださっている家族,父と母に心から感謝します.時々,卒業論文を逃げ道に家事を疎かにしてしまったことを詫びたいと思います.

他にもお世話になった全ての皆様に感謝しつつ,最終版の \LaTeX のコンパイルを実行させていただきます.本当にありがとうございました.

\expandafter\ifx\csname ifdraft\endcsname\relax
  \end{document}
\fi
